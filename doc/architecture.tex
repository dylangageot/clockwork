\documentclass[12pt,landscape]{article}
\usepackage[paperheight=7.25in,paperwidth=16.5in,margin=0.25in]{geometry}

\usepackage{circuitikz}
\usepackage{tikz}
\usetikzlibrary{positioning}
\usetikzlibrary{calc}

\begin{document}

\tikzset{component/.style={font=\small\ttfamily, align=center}}
\tikzset{intercon/.style={circle, fill, inner sep=1.5pt}}
\tikzset{adder/.style={muxdemux, muxdemux def={
            Lh=2, Rh=1, inset w=0.5, inset Lh=1, inset Rh=0.5, w=1.25, NL=2, NB=0, NR=1
        }
    }
}
\tikzset{mux/.style={muxdemux, muxdemux def={
            Lh=4, Rh=2, NL=4, NB=1, NR=1, w=1
        }
    }
}
\tikzset{demux/.style={muxdemux, muxdemux def={
            Lh=2, Rh=4, NL=1, NB=4, NR=1, w=1
        }
    }
}
\tikzset{alu/.style={
        muxdemux, component, muxdemux def={
            Lh=5, Rh=2, NL=2, NR=1, NB=1, NT=0, w=3, 
            inset w=1, inset Lh=2, inset Rh=1, square pins =1
        }
    }
}
\tikzset{register file/.style={
        muxdemux, component, muxdemux def={
            Lh=5, Rh=5, NL=5, NR=5, NB=1, NT=0, w=6.5, square pins =1
        }, no input leads 
    }
}
\tikzset{control unit/.style={
        muxdemux, component, muxdemux def={
            Lh=3, Rh=3, NL=3, NR=1, NB=0, NT=0, w=7, square pins =1
        }, no input leads
    }
}
\tikzset{instruction memory/.style={
        muxdemux, component, muxdemux def={
            Lh=8, Rh=8, NL=5, NR=5, NB=1, NT=0, w=6, square pins =1
        }, no input leads
    }
}
\tikzset{program counter/.style={
        muxdemux, component, muxdemux def={
            Lh=4, Rh=4, NL=3, NR=3, NB=1, NT=0, w=6, square pins =1
        }, no input leads
    }
}
\tikzset{immediate generator/.style={
        muxdemux, component, muxdemux def={
            Lh=2, Rh=2, NL=1, NR=3, NB=0, NT=0, w=4, square pins =1
        }, no input leads
    }
}
\tikzset{memory/.style={
        muxdemux, component, muxdemux def={
            Lh=8, Rh=8, NL=5, NR=5, NB=1, NT=0, w=6, square pins =1
        }, no input leads
    }
}

\begin{circuitikz}[american]
    % thing to know: the alu is the center of the schematic
    \begin{scope}[name=alu]
        \node[alu](alu){\rotatebox{90}{ALU}};
        \draw (alu.bpin 1) node[below, font=\tiny]{$C_{op}$};
    \end{scope}
    
    % rf relative to alu
    \begin{scope}[name=register file]
        % rf
        \node[register file, left=5cm of alu](rf){Register\\File};
        \foreach \rawpin/\label in {1/@rs1, 2/@rs2, 3/@rd, 4/rd} {
            \draw (rf.blpin \rawpin) node[right, font=\tiny]{\label};
        }        
        \draw (rf.blpin 5) ++(0,0.1) -- ++(0.1,-0.1) node[right, font=\tiny]{clk} -- ++(-0.1,-0.1);
        \foreach \rawpin/\label in {2/rs1, 4/rs2} {
            \draw (rf.brpin \rawpin) node[left, font=\tiny]{\label};
        }
        % mux alu port
        \node[mux, above right=1.2cm of rf.rpin 2,
             muxdemux def={NL=2, NT=0, NB=1, Lh=2.5, Rh=1}](mux alu port 1){};
        \node[mux, below right=1.4cm of rf.rpin 4](mux alu port 2){};
        \draw (mux alu port 1.rpin 1) -| (alu.lpin 1);
        \draw (mux alu port 2.rpin 1) -| (alu.lpin 2);
        \draw (mux alu port 1.bpin 1) node[below,font=\tiny]{$C_{amp1}$};
        \draw (mux alu port 2.bpin 1) node[below,font=\tiny]{$C_{amp2}$};

        \draw (rf.bbpin 1) -- ++(0,-0.4) -- ++(0.4,0) node[right, font=\tiny]{$C_{wrd}$};
        \draw (rf.brpin 2) -| (mux alu port 1.lpin 2);
        \draw (rf.brpin 4) -| coordinate[intercon](rs2 con) (mux alu port 2.lpin 1);
    \end{scope}
 
    % cu relative to rf
    \begin{scope}[name=control unit]
        \node[control unit, above right=4cm and 0cm of rf.north west](cu){Control\\Unit};
        \foreach \rawpin/\label in {1/opcode, 2/funct3, 3/funct7} {
            \draw (cu.blpin \rawpin) node[right, font=\tiny]{\label};
        }
        \draw (cu.brpin 1) node[left, align=right, font=\tiny]{control\\vector};
        \draw (cu.brpin 1) -- ++(0.4,0) node[right, font=\tiny]{$C_{...}$};
    \end{scope}

    % immediate
    \begin{scope}[name=immediate generator]
        % immediate i/u/s
        \draw (mux alu port 2.lpin 2) --  (mux alu port 2.lpin 2 -| rf.south east) 
            node[immediate generator, muxdemux def={Lh= 3, Rh = 3}, anchor=brpin 1](ig){Immd I/U/S\\Generator};
        \draw (ig.brpin 1) node[above right, font=\tiny]{I};
        \draw (ig.brpin 2) node[above right, font=\tiny]{U} -- ++(0.6,0)
            coordinate[intercon](u con) node[above right, font=\tiny]{} -- (mux alu port 2.lpin 3);
        \draw (ig.brpin 3) node[above right, font=\tiny]{S} -- (mux alu port 2.lpin 4);
        % immediate j/b
        \node[immediate generator, above right=2.5cm and 0cm of rf.north west](igjb){Immd J/B\\Generator};
        \node[mux, muxdemux def={NL=2, NT=0, NB=1, Lh=2.5, Rh=1}, right=1.1cm of igjb](mux igjb){};
        \draw (igjb.brpin 1) node[above right, font=\tiny]{J} |- (mux igjb.lpin 1);
        \draw (igjb.brpin 3) node[above right, font=\tiny]{B} |- (mux igjb.lpin 2);
        \draw (mux igjb.bpin 1) node[below,font=\tiny]{$C_{jb}$};
        \draw (mux igjb.rpin 1) |- (mux igjb.rpin 1 -| mux alu port 1.blpin 1) 
            node[adder, anchor=blpin 1, no input leads](adder branch){+};
    \end{scope}

    % im relative to rf
    \begin{scope}[name=instruction memory]
        \node[instruction memory, left=3cm of rf](im){Instruction\\Memory};        
        \draw (im.blpin 1) node[right, font=\tiny]{address};
        \draw (im.brpin 1) node[left, font=\tiny]{instruction};
        \draw (im.blpin 5) ++(0,0.1) -- ++(0.1,-0.1) node[right, font=\tiny]{clk} -- ++(-0.1,-0.1);
        %\draw (im.bbpin 1) -- ++(0,-0.4) node[below, font=\tiny]{$C_{epc}$};
    \end{scope}

    % pc relative to im
    \begin{scope}[name=program counter]
        \node[program counter, left=2cm of im.blpin 1, anchor=brpin 1](pc){Program\\Counter};
        \draw (pc.blpin 1) node[right, font=\tiny]{in};
        \draw (pc.brpin 1) node[left, font=\tiny]{out};
        \draw (pc.blpin 3) ++(0,0.1) -- ++(0.1,-0.1) node[right, font=\tiny]{clk} -- ++(-0.1,-0.1);
        % pc input
        \node[mux, muxdemux def={NL=2, NT=1, NB=0, Lh=2.5, Rh=1}, left=1cm of pc.blpin 1](mux pc){};
        \draw (mux pc.rpin 1) -- ++(0,0) -- (pc.blpin 1);
        % adder pc + 4
        \node[adder, left=0.5cm of mux pc.lpin 2](adder pc){+};
        \draw (adder pc.rpin 1) coordinate[](pc 4 con) -- (mux pc.lpin 2);
        \draw (adder pc.lpin 1) node[left]{4};
        \draw (pc.brpin 1) -- ++(1,0) coordinate[intercon](pc out con) |- ++(0,-3) -| (adder pc.lpin 2);
        \draw (pc out con) -- (pc out con |- mux alu port 1.lpin 1) -- ++(0,0.6) -| coordinate[intercon](pc out con 2) (mux alu port 1.lpin 1);
        \draw (pc out con) -- (im.blpin 1);        
        \draw (pc out con 2) coordinate[intercon](mux alu port pc) 
            -- (mux alu port pc |- adder branch.lpin 2) -- (adder branch.blpin 2);
    \end{scope}

    \begin{scope}[name=instruction exploitation]
        \draw[very thick] (im.brpin 1) -- ++(1,0) coordinate[intercon](instruction){};
        \draw[very thick] (instruction.center) -- (instruction |- cu.blpin 1);
        \draw[very thick] (instruction.center) -- (instruction |- ig.blpin 1);
        % to cu
        \draw (cu.blpin 1) -- (cu.blpin 1 -| instruction);
        \draw (cu.blpin 2) -- (cu.blpin 2 -| instruction);
        \draw (cu.blpin 3) -- (cu.blpin 3 -| instruction);
        % to rf
        \draw (rf.blpin 1) -- (rf.blpin 1 -| instruction);
        \draw (rf.blpin 2) -- (rf.blpin 2 -| instruction);
        \draw (rf.blpin 3) -- (rf.blpin 3 -| instruction);
        % to ig
        \draw[very thick] (ig.blpin 1) -- (ig.blpin 1 -| instruction);
        \draw[very thick] (igjb.blpin 1) -- (igjb.blpin 1 -| instruction);
    \end{scope}

    % memory
    \begin{scope}[name=memory]
        \node[memory, right=5cm of alu](mem){Data\\Memory};
        \foreach \rawpin/\label in {1/address, 4/input} {
            \draw (mem.blpin \rawpin) node[right, font=\tiny]{\label};
        }
        \foreach \rawpin/\label in {1/output} { %, 2/ready} {
            \draw (mem.brpin \rawpin) node[left, font=\tiny]{\label};
        }
        \draw (mem.blpin 5) ++(0,0.1) -- ++(0.1,-0.1) node[right, font=\tiny]{clk} -- ++(-0.1,-0.1);
        \draw (alu.rpin 1)-- ++(0.5,0) coordinate[intercon](alu out con) |- coordinate[intercon](alu out con 2) (mem.blpin 1);
        \draw (rs2 con) -| ($(alu.south west) + (-1,-0.8)$) 
            -- ($(alu.south east) + (1.25,-0.8)$) |- (mem.blpin 4);
        \draw (mem.bbpin 1) -- ++(0,-0.4) node[below, font=\tiny]{$C_{wr}$};
    \end{scope}

    % write back mux
    \begin{scope}[name=write back]
        \node[mux, below right= 1cm and 4cm of mem](wb mux){};
        \draw (mem.brpin 1) -- ++(1.5,0) |- (wb mux.lpin 1);
        \draw (alu out con) |- (wb mux.lpin 2);
        \draw (u con) |- (wb mux.lpin 3);
        \draw (wb mux.lpin 4) -- (wb mux.lpin 4 -| pc 4 con) -| (pc 4 con) node[intercon]{};
        \draw (wb mux.rpin 1) -| ++(0.4,-2) -| 
            ($(rf.blpin 4) + (-1,0)$) -- (rf.blpin 4);
        \draw (wb mux.bbpin 1) |- ++(-0.4,-0.6) node[left, font=\tiny]{$C_{wb}$};
    \end{scope}

    % address mux
    \begin{scope}[name=address mux]
        \draw (adder branch.brpin 1) -- (adder branch.brpin 1 -| alu.rpin 1)  
            node[demux, muxdemux def={NR=1, NL=2, NT=0, NB=1, Lh=2.5, Rh=1}, anchor=lpin 1](addr mux){}; 
        \draw (alu out con 2) -- ++(-1,0) |- coordinate[intercon](alu out con 3) (addr mux.lpin 2);
        \draw (addr mux.rpin 1) -- ++(3.5,0) |- ($(cu.north east) + (0,1)$) -| (mux pc.lpin 1);
        \draw (addr mux.bbpin 1) -- ++(0,-0.4) node[below, font=\tiny]{$C_{pc}$};

        \draw (alu out con 3) -- ++(0,1.5) node[xor port, scale=0.75, anchor = in 2](xor branch){};
        \draw (xor branch.out) -- ++(0,0.5) node[or port, scale=0.75, anchor = in 2](or branch){};
        \draw (xor branch.in 1) -- ++(-0.4,0) node[left, font=\tiny]{$C_{neg}$};
        \draw (or branch.in 1) -- ++(-0.4,0) node[left, font=\tiny]{$C_{jump}$};
        \draw (or branch.out) -- ++(2,0) |- ($(cu.north east) + (0,0.5)$) -| (mux pc.tpin 1);
    \end{scope}

    % clk interconnect
    \begin{scope}[name=clk]
        \draw (-23.5,-7) node[left](clk){$clk$} -- (14.5,-7);
        \draw (pc.blpin 3) -- ++(-0.4,0) node[](pc_aux_con){} |- (pc_aux_con |- clk) node[intercon]{};
        \draw (im.blpin 5) -- ++(-0.4,0) node[](aux){} |- (aux |- clk) node[intercon]{};
        \draw (rf.blpin 5) -- ++(-0.4,0) node[](aux){} |- (aux |- clk) node[intercon]{};
        \draw (mem.blpin 5) -- ++(-0.4,0) node[](aux){} |- (aux |- clk) node[intercon]{};
    \end{scope}

    \begin{scope}
        \foreach \x in {-17.7,-10.7,...,14} {
            \draw[dashed] (\x,9) -- (\x,-8);
        }
    \end{scope}

\end{circuitikz}

\end{document}
